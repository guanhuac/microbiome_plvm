\documentclass[oupdraft]{bio}
% \usepackage[colorlinks=true, urlcolor=citecolor, linkcolor=citecolor, citecolor=citecolor]{hyperref}
\usepackage{url}

% Add history information for the article if required
\history{Received August 1, 2010;
revised October 1, 2010;
accepted for publication November 1, 2010}
\usepackage{url}
\usepackage{amssymb, amsfonts, amsmath}
\usepackage{graphicx, natbib}

% ------------------------------------------------------------------------
% Macros
% ------------------------------------------------------------------------
%~~~~~~~~~~~~~~~
% List shorthand
%~~~~~~~~~~~~~~~
\newcommand{\BIT}{\begin{itemize}}
\newcommand{\EIT}{\end{itemize}}
\newcommand{\BNUM}{\begin{enumerate}}
\newcommand{\ENUM}{\end{enumerate}}
%~~~~~~~~~~~~~~~
% Text with quads around it
%~~~~~~~~~~~~~~~
\newcommand{\qtext}[1]{\quad\text{#1}\quad}
%~~~~~~~~~~~~~~~
% Shorthand for math formatting
%~~~~~~~~~~~~~~~
\newcommand\mbb[1]{\mathbb{#1}}
\newcommand\mbf[1]{\mathbf{#1}}
\def\mc#1{\mathcal{#1}}
\def\mrm#1{\mathrm{#1}}
%~~~~~~~~~~~~~~~
% Common sets
%~~~~~~~~~~~~~~~
\def\reals{\mathbb{R}} % Real number symbol
\def\integers{\mathbb{Z}} % Integer symbol
\def\rationals{\mathbb{Q}} % Rational numbers
\def\naturals{\mathbb{N}} % Natural numbers
\def\complex{\mathbb{C}} % Complex numbers
\def\simplex{\mathcal{S}} % Simplex
%~~~~~~~~~~~~~~~
% Common functions
%~~~~~~~~~~~~~~~
\renewcommand{\exp}[1]{\operatorname{exp}\left(#1\right)} % Exponential
\def\indic#1{\mbb{I}\left({#1}\right)} % Indicator function
\providecommand{\argmax}{\mathop\mathrm{arg max}} % Defining math symbols
\providecommand{\argmin}{\mathop\mathrm{arg min}}
\providecommand{\arccos}{\mathop\mathrm{arccos}}
\providecommand{\asinh}{\mathop\mathrm{asinh}}
\providecommand{\dom}{\mathop\mathrm{dom}} % Domain
\providecommand{\range}{\mathop\mathrm{range}} % Range
\providecommand{\diag}{\mathop\mathrm{diag}}
\providecommand{\tr}{\mathop\mathrm{tr}}
\providecommand{\abs}{\mathop\mathrm{abs}}
\providecommand{\card}{\mathop\mathrm{card}}
\providecommand{\sign}{\mathop\mathrm{sign}}
\def\rank#1{\mathrm{rank}({#1})}
\def\supp#1{\mathrm{supp}({#1})}
%~~~~~~~~~~~~~~~
% Common probability symbols
%~~~~~~~~~~~~~~~
\def\E{\mathbb{E}} % Expectation symbol
\def\Earg#1{\E\left[{#1}\right]}
\def\Esubarg#1#2{\E_{#1}\left[{#2}\right]}
\def\P{\mathbb{P}} % Probability symbol
\def\Parg#1{\P\left({#1}\right)}
\def\Psubarg#1#2{\P_{#1}\left[{#2}\right]}
\def\Cov{\mrm{Cov}} % Covariance symbol
\def\Covarg#1{\Cov\left[{#1}\right]}
\def\Covsubarg#1#2{\Cov_{#1}\left[{#2}\right]}
\def\Var{\mrm{Var}}
\def\Vararg#1{\Var\left(#1\right)}
\def\Varsubarg#1#2{\Var_{#1}\left(#2\right)}
\newcommand{\family}{\mathcal{P}} % probability family
\newcommand{\eps}{\epsilon}
\def\absarg#1{\left|#1\right|}
\def\msarg#1{\left(#1\right)^{2}}
\def\logarg#1{\log\left(#1\right)}
%~~~~~~~~~~~~~~~
% Distributions
%~~~~~~~~~~~~~~~
\def\Gsn{\mathcal{N}}
\def\Ber{\textnormal{Ber}}
\def\Bin{\textnormal{Bin}}
\def\Unif{\textnormal{Unif}}
\def\Mult{\textnormal{Mult}}
\def\Cat{\textnormal{Cat}}
\def\Gam{\textnormal{Gam}}
\def\InvGam{\textnormal{InvGam}}
\def\NegMult{\textnormal{NegMult}}
\def\Dir{\textnormal{Dir}}
\def\Lap{\textnormal{Laplace}}
\def\Bet{\textnormal{Beta}}
\def\Poi{\textnormal{Poi}}
\def\HypGeo{\textnormal{HypGeo}}
\def\GEM{\textnormal{GEM}}
\def\BP{\textnormal{BP}}
\def\DP{\textnormal{DP}}
\def\BeP{\textnormal{BeP}}
%~~~~~~~~~~~~~~~
% Theorem-like environments
%~~~~~~~~~~~~~~~

%-----------------------
% Probability sets
%-----------------------
\newcommand{\X}{\mathcal{X}}
\newcommand{\Y}{\mathcal{Y}}
\newcommand{\D}{\mathcal{D}}
\newcommand{\Scal}{\mathcal{S}}
%-----------------------
% vector notation
%-----------------------
\newcommand{\bx}{\mathbf{x}}
\newcommand{\by}{\mathbf{y}}
\newcommand{\bt}{\mathbf{t}}
\newcommand{\xbar}{\overline{x}}
\newcommand{\Xbar}{\overline{X}}
\newcommand{\tolaw}{\xrightarrow{\mathcal{L}}}
\newcommand{\toprob}{\xrightarrow{\mathbb{P}}}
\newcommand{\laweq}{\overset{\mathcal{L}}{=}}
\newcommand{\F}{\mathcal{F}}


\begin{document}

% Title of paper
\title{Latent Variable Modeling for the Microbiome}

\author{
  KRIS SANKARAN$^\ast$, SUSAN HOLMES\\[4pt]
  % Author addresses
  \textit{
    Department of Statistics
    Stanford University
    390 Serra Mall
    Stanford, CA 94305
    United States
  } \\[2pt]
  % E-mail address for correspondence
  {krissankaran@stanford.edu}
}

% Running headers of paper:
\markboth
% First field is the short list of authors
{Sankaran and others}
% Second field is the short title of the paper
{Latent Variable Modeling for the Microbiome}

\maketitle

% Add a footnote for the corresponding author if one has been
% identified in the author list
\footnotetext{To whom correspondence should be addressed.}

\begin{abstract}
  {
    The human microbiome is a complex ecological system, and describing its
    structure and function under different environmental conditions is important
    from both basic scientific and medical perpsectives. From a biostatistical
    point of view, many microbiome analysis goals can be viewed through the lens
    of latent variable modeling. However, although probabilistic latent variable
    models are a cornerstone of modern Bayesian statistics and unsupervised
    learning, they are rarely applied in the context of microbiome data
    analysis, despite the rich structure -- for example, evolutionary, temporal,
    and count structure -- that could be incorporated directly through such
    models. Here, we explore the application of modern probabilistic latent
    variable modeling methods to microbiome data, with a focus on Latent
    Dirichlet Allocation, Nonnegative Matrix Factorization, and Dynamic Unigram
    Models. To develop guidelines for when different methods are appropriate, we
    perform a simulation study. We further illustrate and compare these
    techniques using the data of \citep{dethlefsen2011incomplete}, a study on
    the effects of antibiotics on bacterial community composition. Code and data
    for all simulations and case studies are available publicly.
  }
  {
    Microbiome; Microbial ecology; Latent Dirichlet Allocation; Nonnegative
    Matrix Factorization; Posterior predictive checks; Bayesian data analysis
  }
\end{abstract}

\section{Introduction}

Microbiome studies attempt to characterize variation in bacterial abundance
profiles across different experimental conditions \citep{human2012structure}.
For example, a study may attempt to describe differences in bacterial
communities between diseased and healthy states or after deliberately induced
perturbations \citep{dethlefsen2011incomplete}.

In the process, there tend to be two complementary difficulties. First, the data
are often high-dimensional, measured over several hundreds or thousands of
bacteria. Studying patterns at the level of individual bacteria is typically
uninformative. Second, it can be important to study bacterial abundances in context
of existing biological knowledge. For example, it is scientifically meaningful
when a collection of evolutionarily related bacteria have similar abundance
profiles.

\section{Methods}

We now review a few of the statistical modeling techniques that we believe can
be useful building blocks when performing microbiome analysis. Many of these
techniques have been borrowed from text analysis, thinking of the bacterial
counts matrix as a biological analog of the usual document-term matrix. The idea
of transferring these techniques to the microbiome is not new, though its
appropriateness and usefulness has only been explored in relatively limited
settings \citep{shafiei2015biomico, chen2012estimating, holmes2012dirichlet,
  chen2013variable}.

\subsection{Latent Dirichlet Allocation}

Latent Dirichlet Allocation (LDA) is a generalization of multinomial mixture
modeling applicable to count matrices. We adopt the usual topic modeling
terminology, where each document is summarized by a vector of term counts.
Suppose there are $D$ documents across $V$ terms, and that these documents are
assumed mixtures of $K$ underlying topics, where a topic is defined as a
distribution over words.

Let $\theta_{d} \in \simplex^{K}$ represent the $d^{th}$ topic's mixture over the
$K$ underlying topics. Represent the term in the $n_{th}$ word of this document
by $w_{dn}$, and the associated topic by $z_{dn}$. Suppose the $k^{th}$ topic
places weight $\beta_{vk}$ on the $v^{th}$ term, so that $\beta_{\cdot k} \in
\simplex^{V}$. Then, the generative mechanism is
\begin{align*}
w_{dn} \vert \left(\beta_{\cdot k}\right)_{k = 1}^{K}, z_{dn} &\sim \Mult\left(1, \beta_{\cdot z_{dn}}\right) \\
z_{dn} \vert \theta_{d} &\sim \Mult\left(1, \theta_{d}\right) \\
\theta_{d} &\sim \Dir\left(\alpha\right) \\
\beta_{\cdot k} &\sim \Dir\left(\gamma\right).
\end{align*}
In microbiome applications, we will find a formulation that marginalizes over
the $z_{dn}$ more convenient. Indeed, a strict analogy between text modeling and
microbiome analysis would consider each bacteria a word $w_{dn}$, and we are
more interested in are counts of species across samples. Setting $n_{dv} =
\sum_{n = 1}^{N_{d}} \indic{w_{dn} = v}$, we can write the marginal distribution
as
\begin{align*}
n_{d\cdot} \vert \left(\beta_{k}\right)_{1}^{K} &\sim \Mult\left(n_{d\ast}, \beta \theta_{d}\right) \\
\beta_{\cdot k} &\sim \Dir\left(\gamma\right), k = 1, \dots, K \\
\theta_{d} &\sim \Dir\left(\alpha\right), d = 1, \dots D.
\end{align*}

\subsection{Dynamic Unigram Model}

Geometrically, while LDA identifies samples with points in the convex hull of
$K$ topics on the $V$-dimensional simplex, the Dynamic Unigram Model relates the
change of samples over time with a continuous curve on this $V$-dimensional
simplex \citep{blei2006dynamic}. This curve reflects the gradual evolution of
probabilities over time, and is implemented by passing a random walk through a
multilogit link. That is the Dynamic Unigram Model posits the generative model
\begin{align*}
n_{d\cdot} \vert \mu_{t\left(d\right)}  &\sim \Mult\left(\sum_{v} n_{dv}, S\left(\mu_{t\left(d\right)}\right)\right) \\
\mu_{t} \vert \mu_{t - 1} &\sim \Gsn\left(\mu_{t - 1}, \sigma^{2}I_{V}\right) \\
\mu_{0} &\sim \Gsn\left(0, \sigma^{2}I_{v}\right).
\end{align*}
where $S$ is the multilogit link
\begin{align*}
\left[S\left(\mu\right)\right]_{v} = \frac{\exp{\mu_{v}}}{\sum_{v^{\prime}} \exp{\mu_{v^{\prime}}}},
\end{align*}
and $t\left(d\right)$ maps document $d$ to the time it was sampled. The
$\left(\mu_{t}\right)$ define a Gaussian random walk in $\reals^{V}$ with
stepsize $\sigma$, and $S$ transforms the walk into a sequence of probabilities.

\subsection{Nonnegative Matrix Factorization}
\label{sec:nmf}

In LDA, count matrices are modeled by sampling from a multinomials with total
counts coming from the total number of words in each document and probabilities
coming from the rows of $\Theta B^{T}$ where $\Theta = \begin{pmatrix}\theta_{1}
  \\ \vdots \\ \theta_{D} \end{pmatrix}$ and $B = \begin{pmatrix} \beta_{\cdot 1}
  \dots \beta_{\cdot K} \end{pmatrix}$ are $D \times K$ and $V \times K$ matrices
representing document and topic distributions, respectively, and where each
$\theta_{d} \in S^{K - 1}$ and $\beta_{k} \in S^{V - 1}$.


Alternatively, it is possible to model the nonnegative matrix $N$ by the product
of low rank matrices, $N \approx \Theta B^{T}$, where now the only constraints
on $\Theta$ and $B$ are that $\Theta \in \reals_{+}^{D \times K}$ and $B \in
\reals_{+}^{V \times K}$. This is the starting point for a variety of algorithms
in the Nonnegative Matrix Factorization (NMF) literature
\citep{wang2013nonnegative, berry2007algorithms, lee2001algorithms}

Here, we will consider a Gamma-Poisson factorization model (GaP)
\citep{kucukelbir2015automatic, canny2004gap} which proposes the hierarchical
model
\begin{align*}
N &\sim \Poi\left(\Theta B^{T}\right) \\
\Theta &\sim \Gam\left(a_{0} 1_{D \times K}, b_{0} 1_{D \times K}\right) \\
B &\sim \Gam\left(c_{0} 1_{V \times K}, d_{0} 1_{V \times K} \right),
\end{align*}
where our notation means that each entry in these matrices is sampled
independently, with parameters given by the corresponding entry in the parameter
matrix.

\subsection{Posterior Predictive Checks}
\label{sec:ppc_overview}

Model assessment is important for qualifying interpretations, and can further
guide refinements in subsequent analysis. Indeed, part of the appeal of
probabilistic modeling is the ease with which models can be adapted to better
describe the data of interest. Here, we describe model assessment via posterior
predictive checks \citep{rubin1984bayesianly, gelman2013philosophy}. In this
approach, some statistics $T_{k}\left(x\right)$ of the data are defined which,
in some sense, ``characterize'' the data. If the data $x^{\ast}$ simulated from
the fitted model have statistics $T_{k}\left(x^{\ast}\right)$ with values
similar to those in the observed data $T_{k}\left(x\right)$, then we have
evidence that the proposed model approximates the data well, at least in the
sense defined by $T_{k}$

More precisely, simulate data $x_{1}^{\ast}, \dots x_{S}^{\ast}$ from the
posterior predictive $p\left(x^{\ast}\vert x\right) \approx \int p\left(x^{\ast}
\vert \theta\right) \hat{p}\left(\theta \vert x \right)d\theta$, where $x$ is
the original data and $\hat{p}\left(\theta \vert x\right)$ is an estimate of the
posterior. For each of these simulated data sets, the characterizing statistics
$T_{k}\left(x_{s}\right)$ are computed. Graphically comparing the
$T_{k}\left(x\right)$ calculated on the true data with the histogram of
model-fit simulated $T_{k}\left(x^{\ast}\right)$ suggests ways in which the
posited model fits -- the case where the observed $T_{k}\left(x\right)$ lie in
the bulk of the $T_{k}\left(x^{\ast}_{s}\right)$ -- or fails to fit -- the case
where $T_{k}\left(x\right)$ lie far from the bulk of
$T_{k}\left(x^{\ast}_{s}\right)$ -- the data well.

\subsection{Microbiome vs. Text Analysis}

One of the primary contributions of our work is the observation that methods
popular in text analysis can be adapted to the microbiome setting in a way that
produces useful summaries. Here, we develop the analogy between these
applications, though we also draw attention to points where the parallels break
down.

In the abstract, it becomes clear that the semantic differences between these
fields are often superficial. One mapping between the most common field-specific
terms is provided below,

\begin{itemize}
  \item \textbf{Document} $\iff$ \textbf{Biological Sample}: The basic sampling
    units, over which conclusions are generalized, are documents (text analysis)
    and biological samples (microbiome analysis). It is of interest to highlight
    similarities and differences across these units, often through some
    variation on clustering.
  \item \textbf{Term} $\iff$ \textbf{Bacteria}: The fundamental features
    with which to describe samples are the counts of terms (text analysis) and
    bacteria\footnote{More formally, by bacteria, we mean ribosomal sequence
      variants \citep{callahan2016dada2}. We call these bacteria for simplicity
      of exposition.} (microbiome analysis).
  \item \textbf{Topic} $\iff$ \textbf{Community}: For interpretation, it is
    common to imagine ``typical'' units which can be used as a point of
    reference for observed samples. In text analysis, these are called topics --
    for example, ``business'' or ``politics'' articles have their own specific
    vocabularies. On the other hand, in microbiome analysis, these are called
    ``communities'' -- one common breakdown healthy vs. diseased. Different
    communities have their different bacterial signatures.
  \item \textbf{Word} $\iff$ \textbf{Sequencing Read}: A ``word'' in text analysis refers
    to a single instance of a term in a document, not it's total count. The
    analog in microbiome analysis would be an individual sequencing read, though
    this is rarely the object of statistical analysis.
  \item \textbf{Corpus} $\iff$ \textbf{Environment}: Sometimes a grouping
    structure is known apriori among sampling units. In this case, it can be
    informative to describe whether topics are or are not shared across these
    groups \citep{teh2004sharing}. In the text analysis community, a known group
    of documents -- for example, all articles coming from one newspaper -- is
    called a corpus. In the microbiome community, the associated concept is the
    environment -- for example, skin or soil -- from which a sample was
    obtained.
\end{itemize}

Now that we have established the semantic connections between text and
microbiome analysis, we compare the types of data and analysis goals that are
typical within the respective fields. In both fields, a core unit of analysis is
the sample-by-feature (either document-by-term or biological sample-by-bacteria)
matrix. Besides count structure, the most striking similarity between these data
matrices is sparsity: most entries are zero. Further, observed counts tend to be
highly-skewed -- some terms are far more common than others, and in the same
way, some microbes are much more abundant than others. Finally, in both fields,
contextual information beyond the raw sample-by-feature matrix is typically
available. For example, timestamps are often available in both domains,
$n$-grams have a natural analog in terms of small subnetworks of co-occuring
microbes, and phylogenetic similarity between bacteria parallels a priori known
linguistic characteristics of terms.

Nonetheless, in practice, the structure of these data can be quite different.
First, text data can be on a much larger scale. For example, the Wikipedia
corpus studied in \citep{hoffman2013stochastic} includes 3.8 million articles.
In contrast, even large microbiome data sets typically only have on the order of
hundreds of samples. Similarly, the total number of terms in such large-scale
text analysis problems can be substantially larger than the number of bacteria
species under consideration.

While the limited scale of microbiome studies reduces computational burden, we
note that in one sense, microbiome studies are larger in scale -- there tend to
be tens of thousands of reads per sample in microbiome studies, but only
hundreds to thousands of words within any article. This means that techniques
that rely on the representation of documents as sequences of individual words,
rather than vectors of word counts, require too much memory to be practical.
This makes many useful text analysis techniques -- for model inference
\citep{griffiths2004finding} and evaluation \citep{wallach2009evaluation}, for
example -- out of reach for standard microbiome problems. This does, however,
suggest opporunities for potential future research.

Lastly, we compare the prevailing analytical goals within text and microbiome
analysis. In both fields, data reduction can be informative for developing
models of system behavior. However, an essential difference is that even
unsupervised text analysis techniques are often embedded within automatic
systems, for text classification or information retrieval, for example, which
don't require the intervention of a scientific investigator. In contrast,
in microbiome studies, researchers often have control over specific experimental
design structure.

\section{Simulation Study}

It can be liberating to have easy access to such a variety of modeling
strategies for any given microbiome analysis problem. However, with this
increased flexibility comes the difficulty of determining when to use which
methods. To build some intuition about estimation accuracy across combinations
of data settings and model types, we conduct a series of simulation studies. These are
meant to complement the model-checking that should follow parametric analysis --
since we know the truth in simulations, it is easier to develop more definitive
guidelines.

More specifically, our plan is to divide our analysis into one simulation
generating data from the true LDA model and one drawing from either the NMF or
Z-NMF models. In both, we vary the sample size and dimension. For data simulated
under LDA, we perform model estimation using either Markov Chain Monte Carlo
(MCMC) sampling, Variational Bayes (VB), or a bootstrapping procedure we
describe below, while for the NMF example, we focus on MCMC sampling and VB. The
only misspecification we consider is a failure to account for zero inflation
when the true data were generated according to the Z-NMF model -- though not
pursued here, it could be interesting to study robustness of study conclusions
to misspecification in the number of topics or deliberate contamination.

For the LDA experiment, we vary the number of samples $D \in \{20, 100\}$, the
number of features $V \in \{10, 50\}$, and the total word count per document $N
\in \{20, 50, 100\}$. On the other hand, we fix the number of topics to $K = 2$
and the Dirichlet hyperparameter to $\alpha_{0} = \gamma_{0} = 1$. For each
simulated data set, we perform estimation using VB, MCMC
sampling, and a parametric bootstrap-after-VB procedure. In more detail, this
last parametric bootstrap procedure fits VB to the original data, simulates $B =
500$ new data sets $X^{\ast}_{b}$ according to the LDA model using VB-estimated
parameters $\{\hat{\theta}^{\ast}_{b}, \hat{\beta}^{\ast}_{b}\}$, and
re-estimates parameters $\{\hat{\theta}^{\ast\ast}_{b},
\hat{\beta}^{\ast\ast}_{b}\}$ on each simulated data set, again using VB. The
motivation for this procedure is the desire to strike an easily-parallelizable
compromise between MCMC Sampling, which can be time consuming but has reliable
uncertainty estimates, and VB, which is fast, but can
underestimate uncertainty \citep{wang2005inadequacy}.

Figure \ref{fig:lda_contours} displays the true and posterior $\beta_{k}$ for
the experiments with $V = 10$ features, while varying other simulation characteristics.
Each panel represents a single experimental configuration, with each
axis associated with an underlying topic. Each black number $v$ is the
true value of feature $v$ across topics: $\left(\beta_{v1}, \beta_{v2}\right)$.
The shaded clouds are sampled posteriors, and the orange labels give posterior
means. Across rows, different inferential procedures are compared, the top row of
column labels refers to the total count $N$ within each sample, while the second
refers to the number of samples $D$. The analogous figure when $V = 50$ is
provided in the Supplementary Figure \ref{fig:nmf_v50}.

As expected, when $D$ increases, the posterior for $\beta$, whose dimension does
not increase with $D$, begins to concentrate around its true value. Consistent
with earlier findings, the VB posteriors are narrower and more elliptical than
the more plausible MCMC sampled posteriors. The bootstrap samples seem somewhere
between the VB and MCMC sampled distributions in terms of variability within
each feature. As the number of samples or total count within samples increases,
the three methods become indistinguishable. Interestingly, when $D$ and $N$ are
small ($N = D = 20$), VB appears more accurate than either the MCMC or
bootstrapping approaches, whose posterior means all lie along the diagonal,
corresponding to the failure to estimate distinct topics, and instead defaulting
to marginal feature counts across samples. This effect is especially pronounced
in the case $V = 50$.

For the NMF experiment, we vary the number of features $V$ between $\{75, 125\}$
and zero-inflation probability $p_{0} \in \{0, 0.2\}$. We keep the number of samples
$D$ fixed at $100$ and latent factors $K$ fixed at 2. Unlike LDA, the total
count within each sample is random. Further, in light of the discussion above,
we omit the bootstrap-after-VB estimation technique. Figure
\ref{fig:zinf_thetas} summarizes the estimation error across regimes. The first
row of column names gives $p_{0}$, and the second gives the assumed model: GaP
for the Gamma-Poisson model and Z-GaP for the zero-inflated Gamma-Poisson.
Within each panel, we display histograms of the errors\footnote{We use a
  square-root transformation to ensure the figures are not dominated by very
  large errors.}, $\sqrt{\hat{\theta}_{dk}} - \sqrt{\theta_{dk}}$, where color
encodes the latent factor $k$.

The improved concentration in the high-dimensional ($V = 125$) regime is
more readily apparent. Further, the superiority of MCMC in the
lower-dimensional ($V = 75$) case is suggested by the observation that the MCMC
histogram is somewhat narrower in the center. However, while there seems to be
less mass in the tails of the MCMC histogram, the tails themselves are just as
wide as in the VB histogram.

The most noticeable difference across these panels is that VB
seems to perform substantially worse when zero-inflation is present, especially
when zero-inflation is not taken into account. Surprisingly, MCMC
seems only slightly worse in this situation. Further, the case in which
zero-inflation is not explicitly modeled does not seem to do worse than the case
it is.

\section{Data Analysis}

In applying probabilistic methods to microbiome data analysis, we concentrate on
two questions,
\begin{enumerate}
\item Do these models fit well to the raw or preprocessed data, and what techniques
are available for performing this evaluation?
\item Supposing these models fit well, do they lend themselves to informative
summaries of the original data?
\end{enumerate}

To begin to develop answers to these questions, we reanalyze the data of
\citep{dethlefsen2011incomplete}, a study of bacterial dynamics in response to
antibiotic treatment. This study monitored the microbiomes of three patients
over ten months, with two antibiotics time courses introduced in between, in
order to study the effect of antibiotic perturbations within the context of
natural long-term dynamics. By applying Principal Components Analysis, the study
concluded that antibiotics cause substantial changes in short-term community
composition, with certain species being substantially more resilient than
others, and also detected long-term effects in one patient. The purpose our case
study is to compare these conclusions with those obtained through unsupervised
probabilistic models.

In light of the fact that variation in bacterial signatures tends to be
dominated by strong inter-subject effects, and since with only three subjects,
there is little reason for a model which clusters across subjects, we choose to
study one individual at a time. In this report, we focus on Subject F, who had
been reported to exhibit incomplete recovery of the pre-antibiotic treatment
bacterial community. However, analogous figures for the other two subjects are
available in the supplementary material. Further, we filter to only those
bacteria present in at least 45\% of samples. This reduces the dimensionality
from 2582 to 354.

In this case study, we focus on LDA\footnote{We set $K = 4$, based on the
  heuristic that a larger $K$ would be less meaningful, considering there are
  only 56 timepoints.} and the Dynamic Unigram model. A similar study using GaP
and Z-GaP is omitted for brevity. Throughout, we apply VB, though considering
the results of the simulation study, we exercise caution when interpreting
estimated uncertainties.

Note that we view the fitted probabilities on a logit scale -- for a raw vector
of probabilities $\mathbf{p} = \left(p_{1}, \dots, p_{D}\right)$, we plot
$g\left(\mathbf{p}\right) := \left(\log p_{1} - \overline{\log \mathbf{p}}, \dots,
\log p_{K} - \overline{\log \mathbf{p}}\right)$, which are similar to log-odds,
but centered according to the average log probability, rather than any reference
class.

\subsection{Latent Dirichlet Allocation}
\label{sec:antibiotics_lda}

The fitted parameter values are summarized in Figures
\ref{fig:antibiotics_lda_theta} and \ref{fig:antibiotics_lda_beta}.
In Figure \ref{fig:antibiotics_lda_theta}, rows represent topics, the $x$-axis
represents time, and the $y$-axis gives the transformed fitted probabilities
$g\left(\mathbf{p}\right)$. The boxplots provide posterior quantiles for each
$\theta_{ik}$.

This figure draws attention to the two antibiotic time courses, which took place
between days 12-23 and 41-51. Topic 1 becomes depressed during the time courses,
but recovers during the interim, suggesting that it reflects typical bacterial
community structure. Topic 2 seems to represent those bacteria that were present
initially but are eliminated during the first time course, though there is a
hint of a recovery at the end of sampling. This topic seems most closely related
to the finding reported in \citep{dethlefsen2011incomplete} that Subject F
experienced long-term antibiotic effects on bacterial community composition.
Topics 3 and 4 both seem to be overrepresented during the antibiotic treatments.
Topic 4 is elevated immediately after the cleanout, in contrast, topic 3 seems
to become elevated more gradually. Further, topic 3 appears to be more common in
the samples across all timepoints, including those unassociated with the time
courses. Across all topics, we find that uncertainty is typically smaller for
parameters with larger values.

To interpret these topics in terms of their bacterial community fingerprint, we study the
estimated topic distributions $\beta_{\cdot k}$. This is displayed in Figure
\ref{fig:antibiotics_lda_beta}. The four rows correspond to the $K = 4$
estimated topics. Each boxplot within a row is associated with posterior samples
for a single bacterium. Different colors identify different taxonomic families,
and within these families, bacteria are sorted according to evolutionary
relatedness. We have hidden outliers from the posterior sampling, because they
inflate the $y$-axis range and make the figure harder to read. We have also
removed all but the top four most prevalent taxonomic families, to prevent individual
boxplots from becoming too narrow.

Considering the mixture probabilities in Figure \ref{fig:antibiotics_lda_theta},
those bacteria with large probabilities in the third and fourth rows of Figure
\ref{fig:antibiotics_lda_beta} constitute a large fraction of the samples taken
during antibiotic time courses, reflecting those whose abundances increase
rapidly (Topic 3) or gradually (Topic 4) at the start of antibiotics. This could
be due to these bacteria thriving during the regimen or other bacteria being
more negatively impacted -- viewing the raw data directly supports the latter
scenario. These distributions are relatively more concentrated on a small subset
of bacteria with high probabilities, reflected by the drop in logitted
probabilities far below zero. This corresponds to a decrease in community
diversity during antibiotic time courses.

We also note ``bumps'' of neighboring bacteria with similarly elevated topic
probabilities. It is encouraging that, even without specifying smoothness along
the phylogenetic tree in the prior for the $\beta_{\cdot k}$s, such smoothness
emerges in the fitted model.

\subsection{Dynamic Unigram model}
\label{sec:antibiotics_unigram}

While we can interpret the LDA-estimated $\hat{\theta}_{d}$ according to their
temporal context, this information was never directly provided to the algorithm.
In contrast, we can apply the Dynamic Unigram model to the same data, which
explicitly models temporal evolution. Unlike LDA, however, this model does not
seek latent mixture structure. Our primary results are displayed in Figure
\ref{fig:antibiotics_unigram_theta}.

Each row in this figure is interpreted similarly to a row in Figure
\ref{fig:antibiotics_lda_beta}, except now they correspond to estimated
proportions over time $g\left(\mu_{t}\right)$ rather than transformed topics
$g\left(\beta_{k}\right)$. In particular, only four of the 54 total timepoints
is displayed, highlighting a time window around the first antibiotic time
course. Such a display implicitly assumes a relatively smooth interpolation
between timepoints, which is enforced by the form of the Dynamic Unigram model.

This analysis yields conclusions similar to those obtained through LDA, though
reaching them requires somewhat more effort. For example, on day 10, most
$g\left(\mu_{tv}\right)$'s have most of their mass concentrated above zero, and
few are positioned exceptionally far form the bulk. This is consistent with
higher community diversity before the antibiotic time course. On the other hand,
at time 15, one day after the time course began, most species have quantiles
lower than zero, while a few are positioned much higher than the rest. This
corresponds to a less diverse community, whose membership is concentrated on
those species with outlying boxplots. This decrease in diversity seems most
profound at time 20; though by time 25, during the interim, much of the
community seems to have recovered.

Further, while we continue to see differential recovery across bacteria, the
effect is not as obvious as in LDA, where this effect was decomposed across
topics. For example, many subintervals of Lachnospiraceae seem to return to
their pre-antibiotics levels by time 25, while the Ruminococceae continue to
have low values of $\mu_{tv}$.

\subsection{Posterior Predictive Checks}

While we have found both LDA and the Dynamic Unigram model qualitatively useful,
it is still important to seek more formal diagnostics of model fit. Here, we
consider several posterior predictive cehcks, as foreshadowed in
section \label{sec:ppc_overview}.

To this end, in Figure \ref{fig:antibiotics_posterior_ts}, we plot observed time series for
randomly selected bacteria and contrast them with samples from the posterior
predictive according to the four topic LDA model of
section \ref{sec:antibiotics_lda} and the unigram model of
section \ref{sec:antibiotics_unigram}. Each subpanel corresponds to a single
RSV. The black lines represent observed time series. Note that the $y$-axis scales
vary, as some RSVs are much more abundant than others. Each dot is a simulated
timepoint from a posterior predictive time series.

For LDA, the posterior predictive time series are on the appropriate scale
with approximately the correct shape. However, we observe two substantial types
of departures between simulated and observed data. First, for series with larger
counts, the posterior predictive tends to oversmooth. For example, the drop to 0
in RSV 54 is not captured in any posterior predictive samples. Similar
oversmoothing is visible in RSVs 177, 207, and 263. A more startling type of
departure occurs in the second half of series 207. Here, the posterior
predictive places most mass on the event that the bacterial time series rebounds
to its initial abundance when in reality the species vanishes during the second
antibiotic time course, never to return. A potential explanation for LDA's
failure to capture this pattern is that not many initially highly abundant RSVs
disappear after the second time course, and hence they are not captured by the
global LDA summary. This suggests a technique for highlighting outlier RSVs: we
can look at the average discrepancy between observed series and their posterior
predictive samples.

On the other hand, for the unigram model, the posterior predictive places most
of its support close to the observed RSV series. Usually, this is desirable
behavior, reflecting good model fit. However, here, there is reason for concern
-- the unigram model may not do much more than fit empirical proportions at each
timepoint, and there may be potential to produce more succinct summaries that
still preserve the essential structure of the data. That is, the unigram model
seems overparameterized, simply memorizing the input.

An alternative posterior predictive check compares PCA scores and loadings in
the true and posterior predictive data. Our motivation is that many microbiome
studies base their findings on views generated by PCA, so it would be encouraging
if our probabilistic summaries typically agree with the reductions produced by
PCA.

Figure \ref{fig:antibiotics_posterior_evals} gives the PCA eigenvalues between
the true and posterior predictive samples, after having $\asinh$-transformed
both. The associated scores and loadings figures are available as Supplementary
Figure \ref{fig:antibiotics_posterioc_pca}. The posterior predictive samples
have comparable top four eigenvalues, but rapidly drop-off between the fourth
and fifth eigenvalues. On the other hand, the observed data have a more steady
decline. This is likely a consequence of using $K = 4$ topics in the LDA model,
which would be consistent with the matrix factorization view of LDA described in
section \ref{sec:nmf}. In light of these scree plots, it may be safe to increase
$K$ in follow-up analysis, as long as topics remain interpretable. In contrast,
the eigenvalues for the Dynamic Unigram model closely match those in the
observed data, lending further evidence to the claim that this model is
overparameterized.

\section{Discussion}

We have described the utility of taking a probabilistic modeling perspective in the
analysis of microbiome data. We have provided a detailed implementation of benchmark
analysis approaches, along with exploratory visualization of fitted parameters
and model assessment through posterior predictive checks. Through simulation, we
have established heuristics for determining the appropriateness of applying
different models and inference mechanisms depending on the overall data regime. On a real
microbiome data analysis problem, we have characterized the advantages and
limitations of multiple probabilistic modeling techniques. Rather than focusing
on any single model, like most earlier work, we have emphasized the practice of
contrasting, critiquing, and learning from multiple alternatives. Throughout, we
have emphasized both insights in terms of estimated parameters, as well as
uncertainty, in the form of full approximate posterior distributions.

Microbiome studies are a source of richly structured, high-dimensional data,
coupled with novel scientific problem setups. For example, in the
antibiotics data set described here, we have already encountered structure in
the form of zero-inflated counts, time series with changepoints, and a priori
known phylogenetic relationships between features. More modern microbiome
studies tend to collect more samples as well as more data sources -- spectral
and genomic, in addition bacterial abundance, for example -- per sample.
Further, the investigations often revolve around a combination of
ecological community characterization and medically-relevant identification of
treatment effects. We believe we have only begun to see the potential for
probabilistic methods to guide careful scientific reasoning -- which emphasizes
both insights and the degree of uncertainty about them -- in these richly
structured scenarios.

\section{Supplementary Material}

Supplementary material is available online at
\url{http://biostatistics.oxfordjournals.org}.

\section{Reproducibility}

Code for all simulations, data analysis, and figures is available at
\url{https://github.com/krisrs1128/microbiome_plvm}. Detailed instructions are
available in the repository README.md. Further, a docker image with all software
requirements preinstalled is available from
\url{https://hub.docker.com/r/krisrs1128/microbiome_plvm}.

\section*{Acknowledgments}

{\it Conflict of Interest}: None declared.

\bibliographystyle{biorefs}
\bibliography{refs}

\section{Figures}

\begin{figure}[!p]
  \centering\includegraphics{figure/betacontours1-1}
  \caption{Different inference algorithms for LDA produce different uncertainty
    assessments in small sample sizes, but become comparable as more data arrives.}
  \label{fig:lda_contours}
\end{figure}

\begin{figure}[!p]
  \centering\includegraphics{figure/visualizezinfthetashist-1}
  \caption{Zero inflation poses problems for NMF, even when accounted for in the
    likelihood. The deterioration is most dramatic when applying VB.}
  \label{fig:zinf_thetas}
\end{figure}

\begin{figure}[!p]
  \centering\includegraphics{figure/antibiotics_lda_theta}
  \caption{Boxplots of approximate posteriors for estimated mixture memberships,
    and their evolution over time, suggests a strong effect of the first
    antibiotic treatment, and long term effects, at least for bacteria within
    one topic.}
  \label{fig:antibiotics_lda_theta}
\end{figure}

\begin{figure}[!p]
  \centering\includegraphics{figure/antibiotics_lda_beta}
  \caption{Each boxplot describes an approximate posterior for one $\beta_{vk}$.
    In light of Figure \ref{fig:antibiotics_lda_theta}, this guides the
    interpretation of which bacterial taxa are more or less prevalent during
    antibiotic treatments.}
  \label{fig:antibiotics_lda_beta}
\end{figure}

\begin{figure}[!p]
  \centering
  \includegraphics[scale=1]{figure/antibiotics_unigram_mu}
  \caption{Each boxplot refers to one $\mu_{vt}$. The rows are a subset of times
    $t$ around the first antibiotic time course. This view provides one way of
    smoothing abundance time series, to see how different species respond to
    antibiotic treatment. \label{fig:antibiotics_unigram_theta} }
\end{figure}

\begin{figure}[!p]
  \centering
  \includegraphics[scale=0.2]{figure/posterior_check_evals}
  \caption{As a posterior predictive check, we compute eigenvalues of data
    simulated from the fitted LDA model. The boxplots summarize the posterior
    predictive, while the blue points represent observed data eigenvalues. Note
    that the $y$-axis is on a log scale. Evidently, the four-topic model
    effectively creates a rank-four approximation of the original
    data. \label{fig:antibiotics_posterior_evals}}
\end{figure}

\begin{figure}[!p]
  \centering
  \includegraphics[scale=0.2]{figure/posterior_check_ts}
  \caption{We can visualize the simulated time series for randomly chosen RSVs
    and compare them with the observed ones, as a posterior check.
    \label{fig:antibiotics_posterior_ts}}
\end{figure}


\section{Supplementary Figures}

\begin{figure}[!p]
  \centering
  \includegraphics[scale=0.2]{figure/posterior_check_quantiles}
  \caption{As a posterior check, we compare the observed with simulated overall
    data quantiles. The black, blue, and purple lines represent the observed
    data, posterior samples from LDA, and posterior samples from the Dynamic
    Unigram Model. From this view, we see that the LDA model tends to
    underestimate the overall number of zeros in the data, while the Dynamic
    Unigram Model matches the observed quantiles almost
    exactly. \label{fig:antibiotics_posterior_quantiles} }
\end{figure}


\begin{figure}[!p]
  \centering
  \includegraphics[scale=0.15]{figure/posterior_check_loadings}
  \includegraphics[scale=0.15]{figure/posterior_check_scores}
  \caption{The eigenvalues displayed in Figure
    \ref{fig:antibiotics_posterior_evals} correspond to PCAs computed on
    posterior samples, which are aligned and overlaid here. The left pair of
    panels give loadings for each timepoint, while the right pair provide scores
    for each RSV. The individual posterior samples have been smoothed into
    contours, while the posterior means are displayed as shaded text. The
    observed data PCA results, after alignment with posterior samples, are
    displayed as black text. \label{fig:antibiotics_posterior_pca} }
\end{figure}

\end{document}

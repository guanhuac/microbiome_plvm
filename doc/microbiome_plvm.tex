\documentclass[oupdraft]{bio}
\usepackage{url}
%% \usepackage[colorlinks=true, urlcolor=citecolor, linkcolor=citecolor, citecolor=citecolor]{hyperref}

% Add history information for the article if required
%%\history{Received March 20, 2017}

\begin{document}

% Title of paper
\title{Latent Variable Modeling for the Microbiome} 

% List of authors, with corresponding author marked by asterisk
\author{
  KRIS SANKARAN$^\ast$, SUSAN HOLMES\\[4pt]
  % Author addresses
  \textit{
    Department of Statistics
    Stanford University
    390 Serra Mall
    Stanford, CA 94305
    United States
  } \\[2pt]
  % E-mail address for correspondence
  {krissankaran@stanford.edu}
}

% Running headers of paper:
\markboth
% First field is the short list of authors
{Sankaran and others}
% Second field is the short title of the paper
{Latent Variable Modeling for the Microbiome}

\maketitle

% Add a footnote for the corresponding author if one has been
% identified in the author list
\footnotetext{To whom correspondence should be addressed.}

\begin{abstract}
  {
    The abstract.
  }
  {
    keyword1
  }
\end{abstract}

\section{Introduction}

Generally, microbiome studies attempt to characterize variation in microbial
abundance profiles across different experimental conditions
\cite{human2012structure}. For example, a study may attempt to describe differences
in microbial commuities between diseased and healthy states or after
deliberately induced perturbations \cite{dethlefsen2011incomplete}.

In the process, there tend to be two complementary difficulties. First, the data
are often high dimensional, measured over several hundreds or thousands of
microbes. Studying patterns at the level of individual microbes is typically
infeasible. Second, it can be important to study microbial abundances in context
of known biological information. For example, it is scientifically meaningful
when a collection of evolutionarily related microbes change in sync with one
another.


\section{Methods}

\section{Simulation Study}
\section{Data Analysis}

\section{Discussion}

\section{Software}

\section{Supplementary Material}

Supplementary material is available online at
\url{http://biostatistics.oxfordjournals.org}.

\section*{Acknowledgments}

{\it Conflict of Interest}: None declared.

\bibliographystyle{biorefs}
\bibliography{refs}

\begin{figure}
  \centering
  \includegraphics{figure/betacontours1-1}
  \caption{Different inferenece algorithms for LDA produce different uncertainty
    assessments in small sample sizes, but become comparable as more data arrives.}
  \label{fig:lda_contours}
\end{figure}

\begin{figure}
  \centering
  \includegraphics{figure/visualizezinfthetas-1}
  \caption{Zero inflation poses problems for NMF, even when accounted for in the likelihood. The deterioration is most dramatic when applying VB.}
\end{figure}

\end{document}


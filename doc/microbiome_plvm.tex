\documentclass[oupdraft]{bio}
% \usepackage[colorlinks=true, urlcolor=citecolor, linkcolor=citecolor, citecolor=citecolor]{hyperref}
\usepackage{url}

% Add history information for the article if required
\history{Received August 1, 2010;
revised October 1, 2010;
accepted for publication November 1, 2010}
\usepackage{url}
\usepackage{amssymb, amsfonts, amsmath}
\usepackage{graphicx, natbib}

% ------------------------------------------------------------------------
% Macros
% ------------------------------------------------------------------------
%~~~~~~~~~~~~~~~
% List shorthand
%~~~~~~~~~~~~~~~
\newcommand{\BIT}{\begin{itemize}}
\newcommand{\EIT}{\end{itemize}}
\newcommand{\BNUM}{\begin{enumerate}}
\newcommand{\ENUM}{\end{enumerate}}
%~~~~~~~~~~~~~~~
% Text with quads around it
%~~~~~~~~~~~~~~~
\newcommand{\qtext}[1]{\quad\text{#1}\quad}
%~~~~~~~~~~~~~~~
% Shorthand for math formatting
%~~~~~~~~~~~~~~~
\newcommand\mbb[1]{\mathbb{#1}}
\newcommand\mbf[1]{\mathbf{#1}}
\def\mc#1{\mathcal{#1}}
\def\mrm#1{\mathrm{#1}}
%~~~~~~~~~~~~~~~
% Common sets
%~~~~~~~~~~~~~~~
\def\reals{\mathbb{R}} % Real number symbol
\def\integers{\mathbb{Z}} % Integer symbol
\def\rationals{\mathbb{Q}} % Rational numbers
\def\naturals{\mathbb{N}} % Natural numbers
\def\complex{\mathbb{C}} % Complex numbers
\def\simplex{\mathcal{S}} % Simplex
%~~~~~~~~~~~~~~~
% Common functions
%~~~~~~~~~~~~~~~
\renewcommand{\exp}[1]{\operatorname{exp}\left(#1\right)} % Exponential
\def\indic#1{\mbb{I}\left({#1}\right)} % Indicator function
\providecommand{\argmax}{\mathop\mathrm{arg max}} % Defining math symbols
\providecommand{\argmin}{\mathop\mathrm{arg min}}
\providecommand{\arccos}{\mathop\mathrm{arccos}}
\providecommand{\asinh}{\mathop\mathrm{asinh}}
\providecommand{\dom}{\mathop\mathrm{dom}} % Domain
\providecommand{\range}{\mathop\mathrm{range}} % Range
\providecommand{\diag}{\mathop\mathrm{diag}}
\providecommand{\tr}{\mathop\mathrm{tr}}
\providecommand{\abs}{\mathop\mathrm{abs}}
\providecommand{\card}{\mathop\mathrm{card}}
\providecommand{\sign}{\mathop\mathrm{sign}}
\def\rank#1{\mathrm{rank}({#1})}
\def\supp#1{\mathrm{supp}({#1})}
%~~~~~~~~~~~~~~~
% Common probability symbols
%~~~~~~~~~~~~~~~
\def\E{\mathbb{E}} % Expectation symbol
\def\Earg#1{\E\left[{#1}\right]}
\def\Esubarg#1#2{\E_{#1}\left[{#2}\right]}
\def\P{\mathbb{P}} % Probability symbol
\def\Parg#1{\P\left({#1}\right)}
\def\Psubarg#1#2{\P_{#1}\left[{#2}\right]}
\def\Cov{\mrm{Cov}} % Covariance symbol
\def\Covarg#1{\Cov\left[{#1}\right]}
\def\Covsubarg#1#2{\Cov_{#1}\left[{#2}\right]}
\def\Var{\mrm{Var}}
\def\Vararg#1{\Var\left(#1\right)}
\def\Varsubarg#1#2{\Var_{#1}\left(#2\right)}
\newcommand{\family}{\mathcal{P}} % probability family
\newcommand{\eps}{\epsilon}
\def\absarg#1{\left|#1\right|}
\def\msarg#1{\left(#1\right)^{2}}
\def\logarg#1{\log\left(#1\right)}
%~~~~~~~~~~~~~~~
% Distributions
%~~~~~~~~~~~~~~~
\def\Gsn{\mathcal{N}}
\def\Ber{\textnormal{Ber}}
\def\Bin{\textnormal{Bin}}
\def\Unif{\textnormal{Unif}}
\def\Mult{\textnormal{Mult}}
\def\Cat{\textnormal{Cat}}
\def\Gam{\textnormal{Gam}}
\def\InvGam{\textnormal{InvGam}}
\def\NegMult{\textnormal{NegMult}}
\def\Dir{\textnormal{Dir}}
\def\Lap{\textnormal{Laplace}}
\def\Bet{\textnormal{Beta}}
\def\Poi{\textnormal{Poi}}
\def\HypGeo{\textnormal{HypGeo}}
\def\GEM{\textnormal{GEM}}
\def\BP{\textnormal{BP}}
\def\DP{\textnormal{DP}}
\def\BeP{\textnormal{BeP}}
%~~~~~~~~~~~~~~~
% Theorem-like environments
%~~~~~~~~~~~~~~~

%-----------------------
% Probability sets
%-----------------------
\newcommand{\X}{\mathcal{X}}
\newcommand{\Y}{\mathcal{Y}}
\newcommand{\D}{\mathcal{D}}
\newcommand{\Scal}{\mathcal{S}}
%-----------------------
% vector notation
%-----------------------
\newcommand{\bx}{\mathbf{x}}
\newcommand{\by}{\mathbf{y}}
\newcommand{\bt}{\mathbf{t}}
\newcommand{\xbar}{\overline{x}}
\newcommand{\Xbar}{\overline{X}}
\newcommand{\tolaw}{\xrightarrow{\mathcal{L}}}
\newcommand{\toprob}{\xrightarrow{\mathbb{P}}}
\newcommand{\laweq}{\overset{\mathcal{L}}{=}}
\newcommand{\F}{\mathcal{F}}


\begin{document}

% Title of paper
\title{Latent Variable Modeling for the Microbiome}

\author{
  KRIS SANKARAN$^\ast$, SUSAN HOLMES\\[4pt]
  % Author addresses
  \textit{
    Department of Statistics
    Stanford University
    390 Serra Mall
    Stanford, CA 94305
    United States
  } \\[2pt]
  % E-mail address for correspondence
  {krissankaran@stanford.edu}
}

% Running headers of paper:
\markboth
% First field is the short list of authors
{Sankaran and others}
% Second field is the short title of the paper
{Latent Variable Modeling for the Microbiome}

\maketitle

% Add a footnote for the corresponding author if one has been
% identified in the author list
\footnotetext{To whom correspondence should be addressed.}

\begin{abstract}
  {
    The abstract.
  }
  {
    keyword1
  }
\end{abstract}

\section{Introduction}

Microbiome studies attempt to characterize variation in microbial
abundance profiles across different experimental conditions
\citep{human2012structure}. For example, a study may attempt to describe differences
in microbial commuities between diseased and healthy states or after
deliberately induced perturbations \citep{dethlefsen2011incomplete}.

In the process, there tend to be two complementary difficulties. First, the data
are often high dimensional, measured over several hundreds or thousands of
microbes. Studying patterns at the level of individual microbes is typically
infeasible. Second, it can be important to study microbial abundances in context
of known biological information. For example, it is scientifically meaningful
when a collection of evolutionarily related microbes change in sync with one
another.

\section{Methods}

We now review a few of the statistical modeling techniques that we believe can
be useful building blocks when performing microbiome analysis. Many of these
techniques have been borrowed from text analysis, thinking of the samples by
microbes matrix as a biological analog of the usual document-term matrix. The
idea of transferring these techniques to the microbiome is not new, though its
appropriateness and usefulness has only been explored in relatively limited
settings \citep{shafiei2015biomico, chen2012estimating, holmes2012dirichlet,
  chen2013variable}.

\subsection{Latent Dirichlet Allocation}

Latent Dirichlet Allocation (LDA) is a generalization of multinomial mixture
modeling applicable to count matrices. We adopt the usual topic modeling
terminology, where each document is summarized by a vector of term counts.
Suppose there are $D$ documents across $V$ terms, and that these documents are
assumed mixtures of $K$ underlying topics, where a topic is defined as a
distribution over words.

Let $\theta_{d} \in \simplex^{K}$ represent the $d^{th}$ topic's mixture over the
$K$ underlying topics. Represent the term in the $n_{th}$ word of this document
by $w_{dn}$, and the associated topic by $z_{dn}$. Suppose the $k^{th}$ topic
places weight $\beta_{vk}$ on the $v^{th}$ term, so that $\beta_{\cdot k} \in
\simplex^{V}$. Then, the generative mechanism is

\begin{align*}
w_{dn} \vert \left(\beta_{\cdot k}\right)_{k = 1}^{K}, z_{dn} &\sim \Cat\left(\beta_{\cdot z_{dn}}\right) \\
z_{dn} \vert \theta_{d} &\sim \Cat\left(\theta_{d}\right) \\
\theta_{d} &\sim \Dir\left(\alpha\right) \\
\beta_{\cdot k} &\sim \Dir\left(\gamma\right).
\end{align*}


In the microbiome application, we will find a formulation that marginalizes over
the $z_{dn}$ more convenient. Indeed, a strict analogy between text modeling and
microbiome analysis would consider each living microbe a word $w_{dn}$, while
what we are more interested in are counts of species across samples.
Writing $n_{dv} = \sum_{n = 1}^{N_{d}} \indic{w_{dn} = v}$, we can write the
marginal distribution as

\begin{align*}
n_{d\cdot} \vert \left(\beta_{k}\right)_{1}^{K} &\sim \Mult\left(n_{d\ast}, \beta \theta_{d}\right) \\
\beta_{\cdot k} &\sim \Dir\left(\gamma\right), k = 1, \dots, K \\
\theta_{d} &\sim \Dir\left(\alpha\right), d = 1, \dots D
\end{align*}

\subsection{Dynamic Unigram Model}

While LDA imagines samples being mixtures of fundamental topics on the
$V$-dimensional simplex, the Dynamic Unigram Model identifies a sequence of
samples over time with a curve along this simplex \citep{blei2006dynamic}. To
enforce smoothness in probabilities over over time, a random walk is used is
passed through a softmax function. That is, define,

\begin{align*}
n_{d\cdot} \vert \mu_{t\left(d\right)}  &\sim \Mult\left(\sum_{v} n_{dv}, S\left(\mu_{t\left(d\right)}\right)\right) \\
\mu_{t} \vert \mu_{t - 1} &\sim \Gsn\left(\mu_{t - 1}, \sigma^{2}I_{V}\right) \\
\mu_{0} &\sim \Gsn\left(0, \sigma^{2}I_{v}\right).
\end{align*}
where $S$ is the multilogit link
\begin{align*}
\left[S\left(\mu\right)\right]_{v} = \frac{\exp{mu_{v}}}{\sum_{v^{\prime}} \exp{\mu_{v^{\prime}}}}.
\end{align*}

\subsection{Nonnegative Matrix Factorization}

In LDA, count matrices are modeled by sampling from a multinomials with total
counts coming from the total number of words in each document and probabilities
coming from the rows of $\Theta B^{T}$ where $\Theta = \begin{pmatrix}\theta_{1}
  \\ \vdots \\ \theta_{D} \end{pmatrix} \in \left(\simplex^{K -
  1}\right)_{\downarrow \times D}$ and $B = \begin{pmatrix} \beta_{\cdot 1}
  \dots \beta_{\cdot K} \end{pmatrix} \in \left(\simplex^{V -
  1}\right)_{\rightarrow \times K}$ are $D \times K$ and $V \times K$ matrices
representing document and topic distributions, respectively.


To generalize this idea, it is is possible to model the nonnegative matrix $N$
by the product of low rank matrices, $N \approx \Theta B^{T}$, where now the
only constraints on $\Theta$ and $B$ are that $\Theta \in \reals_{+}^{D \times
  K}$ and $B \in \reals_{+}^{V \times K}$. This is the starting point for a
variety of algorithms in the Nonnegative Matrix Factorization (NMF) literature
\citep{wang2013nonnegative, berry2007algorithms, lee2001algorithms}

Here, we will consider a Gamma-Poisson factorization model (GaP)
\citep{kucukelbir2015automatic, canny2004gap} which is proposes the hierarchical
model
\begin{align*}
N &\sim \Poi\left(\Theta B^{T}\right) \\
\Theta &\sim \Gam\left(a_{0} 1_{D \times K}, b_{0} 1_{D \times K}\right) \\
B &\sim \Gam\left(c_{0} 1_{V \times K}, d_{0} 1_{V \times K} \right),
\end{align*}
where mean that each entry in these matrices is sampled independently, with
parameters given by the corresponding entry in the parameter matrix.

\section{Simulation Study}

It can be liberating to have easy access to such a variety of modeling
strategies for any given microbiome analysis problem. However, with this
increased flexibility comes the difficulty of determining when to use what
methods. To build some intuition about estimation accuracy across combinations
of data settings and model types, we conduct some simulation studies. These are
meant to complement the model-checking that should follow parametric analysis --
since we know the truth in simulations, it is easier to develop more definitive
guidelines.

More specifically, our plan is to divide our analysis into one simulation
generating data from the true LDA model and one drawing from either the NMF or
Z-NMF models. In both, we vary the sample size and dimension. For data simulated
under LDA, we perform model estimation using either MCMC sampling, Variational
Bayes, or a bootstrapping procedure we describe below, while for the NMF
example, we focus on MCMC sampling and Variational Bayes. The only
misspecification we consider is a failure to account for zero inflation when the
true data were generated according to the Z-NMF model -- though not pursued
here, it could be interesting to study robustness of study conclusions to
misspecification in the number of topics or deliberate contamination.

For the LDA experiment, we vary the number of samples $D \in \{20, 100\}$, the
number of features $V \in \{10, 50\}$, and the total word count per document $N
\in \{20, 50, 100\}$. On the other hand, we fix the number of topics to $K = 2$
and the Dirichlet hyperparameter to $\alpha_{0} = \gamma_{0} = 1$. For each
simulated data set, we perform estimation using Variational Bayes, MCMC
sampling, and a parameteric bootstrap-after-VB procedure. In more detail, this
last parametric bootstrap procedure fits VB to the original data, simulates $B =
?$ (todo how many bootstrap replicates?) new data sets $X^{\ast}_{b}$ according
to the LDA model using VB-estimated parameters $\{\hat{\theta}^{\ast}_{b},
\hat{\beta}^{\ast}_{b}\}$, and re-estimates parameters
$\{\hat{\theta}^{\ast\ast}_{b}, \hat{\beta}^{\ast\ast}_{b}\}$ on each simulated data
set, again using VB. The motivation for this procedure is the desire to strike
an easily-parallelizable compromise between MCMC Sampling, which can be time
consuming but has reliable uncertainty estimates, and Variational Bayes, which
is fast, but can underestimate uncertainty.

Figure \ref{fig:lda_contours} displays the true and posterior $\beta_{k}$ for
the experiments with $V = 10$ features, while other simulation characteristics
are varied. Each panel represents a single experimental configuration, with two
axes associated with the two underlying topics. Each black number $v$ is the
true value of feature $v$ across topics: $\left(\beta_{v1}, \beta_{v2}\right)$.
The shaded clouds are sampled posteriors, while the orange labels give posterior
means. Across rows, different inferential procedures are compared The top row of
column labels refers to the total count $N$ within each sample, while the second
refers to the number of samples $D$. The analogous figure when $V = 50$ is
provided in the supplemental material.

As expected, when $D$ increases, the posterior for $\beta$, whose dimension does
not increase with $D$, begins to concentrate around its true value. Consistent
with earlier findings, the VB posteriors are generally lower variance and more
elliptical than the true MCMC sampled posteriors. The bootstrap samples seem
somewhere between the VB and MCMC sampled distributions in terms of variablity
within each feature. As the number of samples or number of words within samples
increases, the three methods become indistinguishable. Interestingly, when $D$
and $N$ are small ($N = D = 20$), VB appears more accurate than either the MCMC
or bootstrapping approaches, whose posterior means all lie along the diagonal,
corresponding to topics that both reflect the global marginal feature counts
across samples. This effect is especially pronounced in the case $V = 50$.

For the NMF experiment, we vary the number of features $V$ between $\{75, 125\}$
and zero-inflation probability $p_{0} \in \{0, 0.2\}$. We keep the number of samples
$D$ fixed at $100$ and latent factors $K$ fixed at 2. Unlike LDA, the total
count within each sample is random. Further, in light of the discussion above,
we omit the bootstrap-after-VB estimation technique. Figure
\ref{fig:zinf_thetas} summarizes the estimation error across regimes. The first
row of column names gives $p_{0}$, and the second gives the assumed model: GaP
for the Gamma-Poisson model and Z-GaP for the zero-inflated Gamma-Poisson.
Within each panel, we display histograms of the errors\footnote{We use a
  square-root transformation to ensure the figures are not dominated by very
  large errors.}, $\sqrt{\hat{theta_{dk}}} - \sqrt{\theta_{dk}}$, where color
encodes the latent factor $k$.

Here, the improved concentration in the high-dimensional ($V = 125$) regime is
more readily apparent. Further, the superiority of gibbs sampling in the
lower-dimensional ($V = 75$) case is still noticeable in the fact that the gibbs
histogram is somewhat narrower in the center. However, while there seems to be
less mass in the tails for the Gibbs histogram, the tails themselves are just as
wide as in the Variational Bayes histogram. 

The most noticeable difference across these panels is that Variational Bayes
seems to perform substantially worse when zero-inflation is present, especially
when zero-inflation is not taken into account. Surprisingly, Gibbs Sampling
seems only slightly worse in this situation. Further, the case in which
zero-inflation is not explicitly modeled does not seem to do worse than the case
it is.

\section{Data Analysis}

In applying probabilistic methods to microbiome data analysis, we concentrate on
two questions,

1. Do these models fit well to the raw or preprocessed data, and what techniques
are available for evaluating model fit?
2. Supposing these models fit well, do they lend themselves to informative
summaries of the original data?

To begin to develop answers to these questions, we reanalyze the data of
\cite{dethlefsen2011incomplete}, a study of microbial dynamics in response to
antibiotic treatment. This study monitored the microbiomes of three patients
over ten months, with two antibiotics time courses introduced in between, in
order to study the effect antiobiotic perturbations within the context of
natural long term dynamics. The study concluded that antibiotics cause
substantial changes in short-term community composition, with certain species
being substantially more resilient than others, and that in one patient,
long-term effects could be observed. The purpose our case study is to compare
these conclusions with those obtained through unsupervised probabilistic models.

In light of the fact that variational in microbial signatures tends to be
dominated by between individual effects, we choose to study one individual at a
time. In this report, we focus on Subject F, who had been reported to exhibit
incomplete recovery of the pre-antibiotic microbial community. Further, we
filter to only those microbes present in at least 45\% of samples. This reduces
the dimensionality from 2582 to 354.

In this case study, we focus on LDA\footnote{We set $K = 4$, based on the
  heuristic that a large $K$ would be less meaningful, considering there are
  only 56 samples.} and the dynamic unigram model. A similar study using GaP and
Z-GaP is omitted for brevity. Throughout, we apply Variational Bayes, though in
light of the simulation study, we exercise caution when interpreting estimated
uncertainties.

Note that we view the fitted probabilities on a logit scale -- for a raw vector
of probabilities $\mathbf{p} = \left(p_{1}, \dots, p_{D}\right)$, we plot
$g\left(\mathbf{p}\right) := \left(\log p_{1} - \bar{\log \mathbf{p}}, \dots,
\log p_{K} - \bar{\log \mathbf{p}}\right)$, which are like log-odds, but
centered according to the average log probability, rather than any reference
class.

The fitted parameter values are summarized in Figures
\@ref(fig:antibiotics_lda_theta), \@ref(fig:antibiotics_lda_beta).
In figure \ref{fig:antibiotics_lda_theta} the rows represent topics, the
$x$-axis represents time, and the $y$-axis gives the fitted probabilties on a
logit scale, but centered according to the average log probability, rather than
any reference class, $g\left(\mathbf{p}\right) = \left(\log p_{1} - \bar{\log
  \mathbf{p}}, \dots, \log p_{K} - \bar{\log \mathbf{p}}\right)$. The boxplots
reflect posterior quantiles.

This figure draws attention to the two antibiotic time courses, which took place
between times 12-23 and 41-51. Topic 1 becomes depressed during the time
courses, but recovers during the interim, suggesting that it summarizes the
"ordinary" microbial community structure. Topic 2 seems to represent those
microbes that were present initially but are eliminated during the first time
course, though there is a hint of a recovery at the very end of sampling. This
topic seems most closely related to the finding reported in
\cite{dethlefsen2011incomplete} that Subject F experienced long-term antibiotic
effects on microbial community composition. Topics 3 and 4 both seem to be
overrepresented during the time courses. Topic 4 is elevated immediately after
the cleanout, in contrast, topic 3 seems to become elevatd more gradually.
Further, topic 3 appears to be more common in the samples across all timepoints,
including those unassociated with the time courses. Across all topics, we find
that uncertainty is typically smaller for parameters with larger values.

\section{Discussion}


\section{Software}

\section{Supplementary Material}

Supplementary material is available online at
\url{http://biostatistics.oxfordjournals.org}.

\section*{Acknowledgments}

{\it Conflict of Interest}: None declared.

\bibliographystyle{biorefs}
\bibliography{refs}

\begin{figure}[!p]
  \centering\includegraphics{figure/betacontours1-1}
  \caption{Different inferenece algorithms for LDA produce different uncertainty
    assessments in small sample sizes, but become comparable as more data arrives.}
  \label{fig:lda_contours}
\end{figure}

\begin{figure}[!p]
  \centering\includegraphics{figure/visualizezinfthetashist-1}
  \caption{Zero inflation poses problems for NMF, even when accounted for in the likelihood. The deterioration is most dramatic when applying VB.}
  \label{fig:zinf_thetas}
\end{figure}

\begin{figure}[!p]
  \centering\includegraphics{figure/antibiotics_theta_boxplot}
  \caption{Boxplots of approximate posteriors for estimated mixture memberships,
    and their evolution over time, suggests a strong effect of the first
    antibiotic treatment, and long term effects, at least for microbes within
    one topic.}
  \label{fig:antibiotics_lda_theta}
\end{figure}
\end{document}

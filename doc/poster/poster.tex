\documentclass[final]{beamer}

\usepackage[scale=1.24]{beamerposter}
\usetheme{confposter}

\newlength{\sepwid}
\newlength{\onecolwid}
\newlength{\twocolwid}
\newlength{\threecolwid}
\setlength{\paperwidth}{48in} % A0 width: 46.8in
\setlength{\paperheight}{36in} % A0 height: 33.1in
\setlength{\sepwid}{0.024\paperwidth}
\setlength{\onecolwid}{0.22\paperwidth}
\setlength{\twocolwid}{0.464\paperwidth}
\setlength{\threecolwid}{0.708\paperwidth}

\usepackage{graphicx}
\usepackage{booktabs}
\usepackage{natbib}


\title{Text Modeling meets the Microbiome}
\author{Kris Sankaran and Susan P. Holmes}
\institute{Department of Statistics, Stanford University}


\begin{document}
\addtobeamertemplate{block end}{}{\vspace*{2ex}}
\addtobeamertemplate{block alerted end}{}{\vspace*{2ex}}

\begin{frame}

\begin{columns}

\begin{column}{\sepwid}\end{column}
\begin{column}{\onecolwid}

\begin{block}{Abstract}
The human microbiome is a complex ecological system, and describing its
structure and function under different environmental conditions is important
from both basic scientific and medical perspectives. Viewed through a
biostatistical lens, many microbiome analysis goals can be approached through
latent variable modeling, for which a range of techniques are available. We
develop the analogy between text modeling, where documents are approximated as
mixtures of topics, and bacterial count modeling, where samples are approximated
as mixtures of ecological states, focusing on applications of Latent Dirichlet
Allocation, Nonnegative Matrix Factorization, and Dynamic Unigrams to the
microbiome. To develop guidelines for when different methods are appropriate, we
perform a simulation study. We further illustrate and compare techniques using
the data of \citep{dethlefsen2011incomplete}, a study on the effects of
antibiotics on bacterial community composition. Code and data for all
simulations and case studies are available publicly.
\end{block}

\begin{figure}[!p]
  %% \centering\includegraphics[width=\textwidth]{../figure/beta_contours_v10}
  \caption{Different inference algorithms for LDA produce different uncertainty
    assessments in small sample sizes, but become comparable as more data arrives.
    Within each panel, we display the true pairs $\left(\sqrt{\beta_{v1}},
    \sqrt{\beta_{v2}}\right)$ as black labels $v$. The purple clouds are
    posterior samples from the inference procedure, which are given as row labels. The
    posterior means are labeled in orange. Different columns index different $D$
    (top column label) vs. $N$ (bottom column label) pairs. Here, we have
    subsetted to $V = 10$ -- the corresponding figure for $V = 50$ is given by
    \ref{fig:beta_contours_v50}.
  }
  \label{fig:beta_contours_v10}
\end{figure}

\begin{figure}[!p]
  \centering
  %% \includegraphics[width=\textwidth]{../figure/beta_errors_lda}
  \caption{A summary of the errors and uncertainty across models and regimes for
    the $\beta_{vk}$ in Figure \ref{fig:beta_contours_v10}. On the $x$-axis,
    we plot the difference between posterior means and the true value, after
    having square-root transformed. On the $y$-axis, we provide the standard
    deviation of the posterior samples for each $v$, along the first dimension
    $K = 1$. Columns are indexed as in Figure \ref{fig:beta_contours_v10}, but
    now rows provide different values of $V$.
    \label{fig:beta_errors_lda} }
\end{figure}

\end{column} 

\begin{column}{\sepwid}\end{column}
\end{columns}
\begin{column}

\section{Figures}
\begin{figure}[!p]
  \centering\includegraphics{../figure/visualize_lda_theta_boxplot-F}
  \caption{Boxplots represent approximate posteriors for estimated mixture
    memberships $\theta_{d}$, and their evolution over time. That is, each row
    of panels provides a different sequence of $\theta_{dk}$ for a single $k$,
    and different columns distinguish different phases of sampling. Note that
    the $y$-axis is on the $g$-scale, which is defined as a translated logit,
    $g\left(\mathbf{p}\right) := \left(\log p_{1} - \bar{\log \mathbf{p}},
    \dots,\log p_{K} - \bar{\log \mathbf{p}}\right)$. Note that the first
    and second antibiotic time courses result in meaningful shifts in these
    sequences, and that there appear to be long-term effects of treatment among
    bacteria in Topic 3. \label{fig:antibiotics_lda_theta}}
\end{figure}
\end{column}

\end{frame}
\bibliographystyle{plainnat}
\bibliography{../refs.bib}

\end{document}

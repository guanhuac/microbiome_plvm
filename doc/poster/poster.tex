\documentclass[final, 8pt]{beamer}

\usepackage[size=custom,width=76.2,height=101.6,scale=0.85]{beamerposter}
\usetheme{confposter}

\newlength{\sepwid}
\newlength{\onecolwid}
\newlength{\twocolwid}
\newlength{\threecolwid}
\setlength{\paperwidth}{30in} % A0 width: 46.8in
\setlength{\paperheight}{40in} % A0 height: 33.1in
\setlength{\onecolwid}{0.3\paperwidth}
\setlength{\twocolwid}{0.464\paperwidth}
\setlength{\threecolwid}{0.708\paperwidth}
\setbeamersize{text margin left=0.05cm,text margin right=0.05cm}

\usepackage{graphicx}
\usepackage{booktabs}
\usepackage{natbib}
\usepackage{url}
\usepackage{amssymb, amsfonts, amsmath}
\usepackage{graphicx, natbib}

% ------------------------------------------------------------------------
% Macros
% ------------------------------------------------------------------------
%~~~~~~~~~~~~~~~
% List shorthand
%~~~~~~~~~~~~~~~
\newcommand{\BIT}{\begin{itemize}}
\newcommand{\EIT}{\end{itemize}}
\newcommand{\BNUM}{\begin{enumerate}}
\newcommand{\ENUM}{\end{enumerate}}
%~~~~~~~~~~~~~~~
% Text with quads around it
%~~~~~~~~~~~~~~~
\newcommand{\qtext}[1]{\quad\text{#1}\quad}
%~~~~~~~~~~~~~~~
% Shorthand for math formatting
%~~~~~~~~~~~~~~~
\newcommand\mbb[1]{\mathbb{#1}}
\newcommand\mbf[1]{\mathbf{#1}}
\def\mc#1{\mathcal{#1}}
\def\mrm#1{\mathrm{#1}}
%~~~~~~~~~~~~~~~
% Common sets
%~~~~~~~~~~~~~~~
\def\reals{\mathbb{R}} % Real number symbol
\def\integers{\mathbb{Z}} % Integer symbol
\def\rationals{\mathbb{Q}} % Rational numbers
\def\naturals{\mathbb{N}} % Natural numbers
\def\complex{\mathbb{C}} % Complex numbers
\def\simplex{\mathcal{S}} % Simplex
%~~~~~~~~~~~~~~~
% Common functions
%~~~~~~~~~~~~~~~
\renewcommand{\exp}[1]{\operatorname{exp}\left(#1\right)} % Exponential
\def\indic#1{\mbb{I}\left({#1}\right)} % Indicator function
\providecommand{\argmax}{\mathop\mathrm{arg max}} % Defining math symbols
\providecommand{\argmin}{\mathop\mathrm{arg min}}
\providecommand{\arccos}{\mathop\mathrm{arccos}}
\providecommand{\asinh}{\mathop\mathrm{asinh}}
\providecommand{\dom}{\mathop\mathrm{dom}} % Domain
\providecommand{\range}{\mathop\mathrm{range}} % Range
\providecommand{\diag}{\mathop\mathrm{diag}}
\providecommand{\tr}{\mathop\mathrm{tr}}
\providecommand{\abs}{\mathop\mathrm{abs}}
\providecommand{\card}{\mathop\mathrm{card}}
\providecommand{\sign}{\mathop\mathrm{sign}}
\def\rank#1{\mathrm{rank}({#1})}
\def\supp#1{\mathrm{supp}({#1})}
%~~~~~~~~~~~~~~~
% Common probability symbols
%~~~~~~~~~~~~~~~
\def\E{\mathbb{E}} % Expectation symbol
\def\Earg#1{\E\left[{#1}\right]}
\def\Esubarg#1#2{\E_{#1}\left[{#2}\right]}
\def\P{\mathbb{P}} % Probability symbol
\def\Parg#1{\P\left({#1}\right)}
\def\Psubarg#1#2{\P_{#1}\left[{#2}\right]}
\def\Cov{\mrm{Cov}} % Covariance symbol
\def\Covarg#1{\Cov\left[{#1}\right]}
\def\Covsubarg#1#2{\Cov_{#1}\left[{#2}\right]}
\def\Var{\mrm{Var}}
\def\Vararg#1{\Var\left(#1\right)}
\def\Varsubarg#1#2{\Var_{#1}\left(#2\right)}
\newcommand{\family}{\mathcal{P}} % probability family
\newcommand{\eps}{\epsilon}
\def\absarg#1{\left|#1\right|}
\def\msarg#1{\left(#1\right)^{2}}
\def\logarg#1{\log\left(#1\right)}
%~~~~~~~~~~~~~~~
% Distributions
%~~~~~~~~~~~~~~~
\def\Gsn{\mathcal{N}}
\def\Ber{\textnormal{Ber}}
\def\Bin{\textnormal{Bin}}
\def\Unif{\textnormal{Unif}}
\def\Mult{\textnormal{Mult}}
\def\Cat{\textnormal{Cat}}
\def\Gam{\textnormal{Gam}}
\def\InvGam{\textnormal{InvGam}}
\def\NegMult{\textnormal{NegMult}}
\def\Dir{\textnormal{Dir}}
\def\Lap{\textnormal{Laplace}}
\def\Bet{\textnormal{Beta}}
\def\Poi{\textnormal{Poi}}
\def\HypGeo{\textnormal{HypGeo}}
\def\GEM{\textnormal{GEM}}
\def\BP{\textnormal{BP}}
\def\DP{\textnormal{DP}}
\def\BeP{\textnormal{BeP}}
%~~~~~~~~~~~~~~~
% Theorem-like environments
%~~~~~~~~~~~~~~~

%-----------------------
% Probability sets
%-----------------------
\newcommand{\X}{\mathcal{X}}
\newcommand{\Y}{\mathcal{Y}}
\newcommand{\D}{\mathcal{D}}
\newcommand{\Scal}{\mathcal{S}}
%-----------------------
% vector notation
%-----------------------
\newcommand{\bx}{\mathbf{x}}
\newcommand{\by}{\mathbf{y}}
\newcommand{\bt}{\mathbf{t}}
\newcommand{\xbar}{\overline{x}}
\newcommand{\Xbar}{\overline{X}}
\newcommand{\tolaw}{\xrightarrow{\mathcal{L}}}
\newcommand{\toprob}{\xrightarrow{\mathbb{P}}}
\newcommand{\laweq}{\overset{\mathcal{L}}{=}}
\newcommand{\F}{\mathcal{F}}


\title{Text Modeling meets the Microbiome}
\author{Kris Sankaran and Susan P. Holmes}
\institute{Department of Statistics, Stanford University}

\begin{document}

\begin{frame}

\begin{columns}
\begin{column}{\onecolwid}

\begin{block}{Abstract}
The human microbiome is a complex ecological system, and describing its
structure and function under different environmental conditions is important
from both basic scientific and medical perspectives. Viewed through a
biostatistical lens, many microbiome analysis goals can be approached through
latent variable modeling, for which a range of techniques are available. We
develop the analogy between text modeling, where documents are approximated as
mixtures of topics, and bacterial count modeling, where samples are approximated
as mixtures of ecological states, focusing on applications of Latent Dirichlet
Allocation (LDA), Nonnegative Matrix Factorization (NMF), and Dynamic Unigrams
to the microbiome. We further illustrate and compare techniques using the data
of \citep{dethlefsen2011incomplete}, a study on the effects of antibiotics on
bacterial community composition. A complete preprint is available
\citep{sankaran2017latent}, and code and data for reproducing model estimates
and figures can be found at
\url{https://github.com/krisrs1128/microbiome_plvm/}.
\end{block}

\begin{block}{Methods}
  \begin{itemize}
\item LDA: Let $x_{dv}$ be the number of times word $v$
  occurs in document $d$. Suppose the $k^{th}$ topic places weight $\beta_{vk}$
  on the $v^{th}$ term, so that $\beta_{k} \in \simplex^{V - 1}$. Suppose there
  are $N_{d}$ words in the $d^{th}$ document. Then,
\begin{align*}
x_{d\cdot} \vert \left(\beta_{k}\right)_{1}^{K} &\overset{iid}{\sim} \Mult\left(N_{d}, B\theta_{d}\right) \text{ for } d = 1, \dots, D\\
\theta_{d} &\overset{iid}{\sim} \Dir\left(\alpha\right) \text{ for } d = 1, \dots, D \\
\beta_{k} &\overset{iid}{\sim} \Dir\left(\gamma\right) \text{ for }k = 1, \dots, K,
\end{align*}
where $B = \begin{pmatrix}\beta_{1}, \dots, \beta_{K}\end{pmatrix}$.

\item Dynamic Unigrams: In light of LDA's geometric interpretation, we
  might consider in some situations a model that identifies samples with a
  continuous curve on this $V$-dimensional simplex \citep{blei2006dynamic},
\begin{align*}
x_{d\cdot} \vert \mu_{t\left(d\right)}  &\overset{iid}{\sim} \Mult\left(N_{d}, S\left(\mu_{t\left(d\right)}\right)\right) \text{ for } d = 1, \dots, D\\
\mu_{t} \vert \mu_{t - 1} &\overset{iid}{\sim} \Gsn\left(\mu_{t - 1}, \sigma^{2}I_{V}\right) \text{ for } t = 1, \dots, T \\
\mu_{0} &\overset{iid}{\sim} \Gsn\left(0, \sigma^{2}I_{V}\right),
\end{align*}
where $S$ is the multilogit link
\begin{align*}
\left[S\left(\mu\right)\right]_{v} = \frac{\exp{\mu_{v}}}{\sum_{v^{\prime}} \exp{\mu_{v^{\prime}}}},
\end{align*}
and $t\left(d\right)$ maps document $d$ to the time it was sampled.
\item NMF: We consider a Gamma-Poisson factorization model
  \citep{zhou2015negative} that models the $D \times V$ counts matrix $X$ by
  \begin{align*}
    X &\sim \Poi\left(\Theta B^{T}\right) \\
    \Theta &\sim \Gam\left(a_{0} 1_{D \times K}, b_{0} 1_{D \times K}\right) \\
    B &\sim \Gam\left(c_{0} 1_{V \times K}, d_{0} 1_{V \times K} \right),
  \end{align*}
  where our notation means that each entry in these matrices is sampled
  independently, with parameters given by the corresponding entry in the parameter
  matrix. In our simulation studies, we also consider a slight variant, which
  independently sends entries of $X$ to zero with probability $p_{0}$.
\end{itemize}
\end{block}

\begin{block}{Microbiome vs. Text Analysis}
Methods popular in text analysis can be adapted to the microbiome setting, upon
making the following connections.

\begin{itemize}
  \item \textbf{Document} $\iff$ \textbf{Biological Sample}: The basic sampling
    units, over which conclusions are generalized, are documents (text analysis)
    and biological samples (microbiome analysis). It is of interest to highlight
    similarities and differences across these units, often through some
    variation on clustering or dimensionality reduction.
  \item \textbf{Term} $\iff$ \textbf{Bacterial species}: The fundamental
    features with which to describe samples are the counts of terms (text
    analysis) and bacterial species (microbiome analysis). More formally, by
    bacterial species, we mean Amplicon Sequence Variants
    \citep{callahan2017exact}.
  \item \textbf{Topic} $\iff$ \textbf{Community}: For interpretation, it is
    common to imagine ``prototypical'' units which can be used as a point of
    reference for observed samples. In text analysis, these are called topics --
    for example, ``business'' or ``politics'' articles have their own specific
    vocabularies. On the other hand, in microbiome analysis, these are called
    ``communities'' -- different communities have different bacterial
    signatures.
  \item \textbf{Word} $\iff$ \textbf{Sequencing Read}: A ``word'' in text analysis refers
    to a single instance of a term in a document, not its total count. The
    analog in microbiome analysis is an individual read that has been mapped to
    a unique sequence variant, though this is rarely an object of intrinsic interest.
\end{itemize}
\end{block}

\end{column}

\begin{column}{\onecolwid}

\begin{block}{Case Study}
We reanalyze the data of \cite{dethlefsen2011incomplete}, a study of bacterial
dynamics in response to antibiotic treatment. This study monitored three
patients over time, with two antibiotics time courses introduced within small
windows, in order to study the effect of antibiotic perturbations within the
context of natural long-term dynamics.

Variation in bacterial signatures tends to be dominated by strong inter-subject
effects \citep{eckburg2005diversity}, and with only three subjects, there is
little reason to cluster across subjects. Instead, we focus on Subject F, who
had been reported to exhibit incomplete recovery of the pre-antibiotic treatment
bacterial community.

\end{block}

\begin{figure}[!p]
  \centering\includegraphics[width=\textwidth]{../figure/visualize_lda_theta_boxplot-F}
  \caption{Boxplots represent approximate posteriors for estimated mixture
    memberships $\theta_{d}$, and their evolution over time. Each row of panels
    provides a different sequence of $\theta_{dk}$ for a single $k$, and
    different columns distinguish different phases of sampling. Note that the
    $y$-axis is on the $g$-scale, which is defined as a translated logit,
    $g\left(\mathbf{p}\right) := \left(\log p_{1} - \overline{\log \mathbf{p}},
    \dots,\log p_{K} - \overline{\log \mathbf{p}}\right)$. The first and second
    antibiotic time courses result in meaningful shifts in these sequences, and
    that there appear to be long-term effects of treatment among bacteria in
    Topic 3. \label{fig:antibiotics_lda_theta}}
\end{figure}

%% To interpret topics in terms of their bacterial community fingerprints, we study
%% the $\beta_{k}$, displayed in Figure \ref{fig:antibiotics_lda_beta}. Those
%% bacteria with large probabilities in the second and fourth rows of Figure
%% constitute a large fraction of the samples taken during antibiotics time
%% courses, reflecting those whose abundances increase rapidly (Topic 2) or are
%% initially drop but then increase gradually (Topic 4) during the antibiotics time
%% courses. This could be due to these bacteria
\begin{block}{}
Figure \ref{fig:antibiotics_lda_theta} draws attention to the two antibiotic
time courses, which took place between days 12-23 and 41-51. Topic 1 appears to
represent the bacterial community filling niches left empty during the first
time course, 3 are those that fail to recover after time course 1, while 2 and 4
reflect those that thrive during antibiotic time courses, with different
response times. To interpret topics in terms of their bacterial community
fingerprints, we display the $\beta_{k}$ in Figure
\ref{fig:antibiotics_lda_beta}.

Figure \ref{fig:antibiotics_unigram_theta} displays a subset of the results of
the alternative Dynamic Unigrams approach instead. Differential responses to
antibiotic treatment is reflected in the differential change in $\mu_{tv}$
across different $v$'s, as time is varied.
\end{block}

\begin{figure}[!p]
  \centering\includegraphics[width=1.01\textwidth]{../figure/visualize_lda_beta-F}
  \caption{Each credible interval describes an approximate posterior for one
    $\beta_{vk}$. Coupled with Figure \ref{fig:antibiotics_lda_theta}, this
    guides the interpretation of which bacterial taxa are more or less prevalent
    during antibiotic treatments. The $x$-axis indexes species, sorted according
    to phylogenetic relatedness, and the $y$-axis give transformed values of the
    species probability under that topic. Only the 750 most abundant species are
    shown. Note the disappearance of otherwise abundant species within Topics 2,
    4, and to some extent, 1.} \label{fig:antibiotics_lda_beta}
\end{figure}

\begin{figure}[ht]
  \centering
  \includegraphics[width=1.01\textwidth]{../figure/antibiotics_unigram_mu-F}
  \caption{Each posterior credible interval refers to one $\mu_{vt}$. The rows
    are a subset of times $t$ around the first antibiotic time course. This
    display is read in the same way as Figure \ref{fig:antibiotics_lda_beta}.
    This view provides one way of smoothing abundance time series, to see how
    different species respond to antibiotic
    treatment. \label{fig:antibiotics_unigram_theta} }
\end{figure}


\end{column}

\begin{column}{\onecolwid}

\begin{block}{Posterior Predictive Checks}
Model assessment is important for qualifying interpretations, and can further
guide refinements in subsequent analyses. Two example checks are provided in
Figures \ref{fig:antibiotics_posterior_ts} and
\ref{fig:antibiotics_posterior_pca}. For LDA, the posterior predictive time
series are on the appropriate scale with approximately the correct shape.
However, for series with larger counts, the posterior predictive tends to
oversmooth. For example, the drop to 0 in species 343 is not captured in any
posterior predictive samples.

On the other hand, for the Dynamic Unigram model, the posterior predictive
distribution places most of its support close to the observed species series.
This is reason for concern -- there may be potential to produce more succinct
summaries that still preserve the essential structure of the data.

\end{block}

\begin{figure}[!p]
  \centering
  \includegraphics[width=\textwidth]{../figure/posterior_check_ts-F}
  \caption{We can visualize simulated time series for a subset of species and
    compare them with the observed ones, as a posterior check. Each panel
    represents one species. The black lines give observed $\asinh$-transformed
    abundances for subject F over time. The blue and purple dots give posterior
    predictive realizations for species over time, according to LDA and Dynamic
    Unigrams, respectively.
    \label{fig:antibiotics_posterior_ts}}
\end{figure}

\begin{figure}[!p]
  \centering
  \includegraphics[width=\textwidth]{../figure/posterior_check_scores-loadings-F}
  \caption{
    The PCA results computed on posterior predictive samples are aligned and
    overlaid here. The left pair of panels give scores for each species, while
    the right pair provide loadings for each timepoint. The individual posterior
    samples have been smoothed into contours, while the posterior means are
    displayed as shaded text. The observed data PCA results, after alignment
    with posterior samples, are displayed as black
    text. \label{fig:antibiotics_posterior_pca} }
\end{figure}


\begin{block}{References}
\bibliographystyle{plainnat}
\bibliography{../refs.bib}
\end{block}

\end{column}
\end{columns}


\end{frame}
\end{document}
